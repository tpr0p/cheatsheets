\documentclass[12pt]{article}

% imports
\usepackage{appendix}
\usepackage{amsmath}
\usepackage{amssymb}
\usepackage[english]{babel}
\usepackage{bbold}
\usepackage[justification=centering]{caption}
\usepackage{circuitikz}
\usepackage{float}
\usepackage[margin=1em]{geometry}
\usepackage{graphicx}
\usepackage{hyperref}
\usepackage[utf8]{inputenc}
\usepackage{layouts}
\usepackage{multirow}
\usepackage{multicol}
\usepackage{optidef}
\usepackage{physics}
\usepackage{setspace}
\usepackage{diagbox}
\usepackage[labelformat=simple]{subcaption}
\usepackage{xcolor}

% configure imports
\DeclareMathOperator*{\argmin}{arg\,min}
\newcommand{\todo}[1]{\textcolor{red}{TODO: #1}}
\definecolor{darkgreen}{RGB}{46, 184, 46}
\newcommand{\half}{\frac{1}{2}}
\newcommand{\R}{\mathbb{R}}
\captionsetup[subfigure]{labelformat=empty, skip=0pt}
\captionsetup{labelsep=period}
\restylefloat{table}
\renewcommand\thesubfigure{(\alph{subfigure})}
\addto\captionsenglish{%
  \renewcommand{\figurename}{FIG.}%
  \renewcommand{\tablename}{TABLE}%
  \renewcommand{\appendixname}{APPENDIX}%
}

\title{\vspace{-2em}Circuits}
\date{}

\begin{document}

\maketitle

\vspace{-4em}
\begin{multicols}{2}
  \textbf{Inductors, Capacitors, Resistors}
  \begin{align}
    \sum_{\textrm{path $i$ to common node}} I_i &= 0\\
    \sum_{\textrm{device $i$ on closed loop}} V_i &= 0\\
    V_R &= I R\\
    V_L &= L \dot{I}\\
    V_C &= \frac{1}{C} \int_t I\\
    R_{\textrm{series}} &= \sum_i R_i\\
    1/R_{\textrm{parallel}} &= \sum_i 1/R_i\\
    L_{\textrm{series}} &= \sum_i L_i\\
    1/L_{\textrm{parallel}} &= \sum_i 1/L_i\\
    1/C_{\textrm{series}} &= \sum_i 1/C_i\\
    C_{\textrm{parallel}} &= \sum_i C_i\\
  \end{align}

  \textbf{Diode Axioms}
  \begin{itemize}
  \item no negative current when on
  \item no positive voltage when off
  \end{itemize}

  \textbf{Periodic Steady State}
  \begin{itemize}
    \item $\langle v_L \rangle = 0$
    \item $\langle i_C \rangle = 0$
  \end{itemize}

  \textbf{Transformer}
  \begin{align}
    \frac{v_1}{v_2} &= \frac{N_1}{N_2}\\ %% KPVS eq 1.1
    \frac{i_1}{i_2} &= \frac{-N_2}{N_1} %% KPVS eq 1.2
  \end{align}

  \textbf{MOS}
  \begin{align}
    V_T &= V_{T0} + \gamma (\sqrt{\lvert -2\phi_F + V_{SB}\rvert} - \sqrt{\lvert -2\phi_F \rvert})\\
    I_D &= \begin{cases}
      k[V_{GT}V_{\textrm{min}} - V_{\textrm{min}}^2/2](1 + \lambda V_{DS}), \quad V_{GT} \geq 0\\
      0, \qquad \qquad \qquad \qquad \qquad \qquad \qquad \quad \; V_{GT} \leq 0
    \end{cases}\\ % Rabaey eq 3.25, 3.29, 3.30, 3.39
    V_{\textrm{min}} &= \textrm{min}(V_{DS}, V_{DSAT}, V_{GT}) \nonumber\\
    V_{GT} &= V_{GS} - V_T \nonumber\\
    k &= k^{\prime} W / L \nonumber\\
    k^{\prime} &= \mu \epsilon_{\textrm{oxide}} / t_{\textrm{oxide}} \nonumber\\ % eq 3.19
    \mu &= e \tau / m^{*} \nonumber\\
    R_{eq} &\approx \frac{3}{4} \frac{V_{DD}}{I_{DSAT}}(1 - \frac{7}{9} \lambda V_{DD})\\ % eq 5.17
    P_{\textrm{dyn}} &= C_L V_{DD}^2 f_{0 \rightarrow 1}\\ % eq 5.42
    P_{\textrm{leak}} &= I_{\textrm{leak}} V_{DD}
  \end{align}

  \textbf{Inverter}
  \begin{align}
    V_M &= \begin{aligned}
      &[(V_{Tn} + V_{DSATn}/2) + r (V_{DD} + V_{Tp} + V_{DSATp}/2)]\\ % Rabaey eq 5.3
      &/ (1 + r)
    \end{aligned}\\
    r &= \frac{k_p V_{DSATp}}{k_n V_{DSATn}} \nonumber\\
    t_p &= \int_{V_1}^{V_2} \frac{C_L(V)}{I(V)} dV\\ % 5.16
    t_{pHL} &= \ln(2)R_{eqn}C_L\\ % 5.18
    t_{pLH} &= \ln(2)R_{eqp}C_L\\ % 5.19
    NM_H &= V_{DD} - V_{IH}\\ %  eq 5.7
    NM_L &= V_{IL}\\
    V_{IH}, &V_{IL} \sim \partial V_{out}/\partial V_{in} = -1\\
    P_{DD}(t) &= I_{DD}(t) \cdot  V_{DD}\\ % eq
    PDP &= \langle P \rangle t_p % 5.53
  \end{align}

  \textbf{Symbols}
  \\
  \begin{circuitikz}
    \draw (0, 0) node [ieeestd and port] {AND};
    \draw (0, -1.5) node [ieeestd buffer port] {BUF};
    \draw (3, 0) node [ieeestd or port] {OR};
    \draw (3, -1.5) node [ieeestd xor port] {XOR};
  \end{circuitikz}

  \textbf{Misc}
  \begin{align}
    e &= 10^{x/10 \textrm{dB}}\\
    x &= 10 \ \textrm{log}_{10}(e) \ \textrm{dB}\\
    a \ \textrm{sin} (x) + b \ \textrm{cos} (x) &= \sqrt{a^2 + b^2} \ \textrm{sin}[x + \textrm{atan} \ (b/a)]
  \end{align}
\end{multicols}



\end{document}
