\documentclass[12pt]{article}

% imports
\usepackage{appendix}
\usepackage{amsmath}
\usepackage{amssymb}
\usepackage[english]{babel}
\usepackage{bbold}
\usepackage[justification=centering]{caption}
\usepackage{float}
\usepackage[margin=1em]{geometry}
\usepackage{graphicx}
\usepackage{hyperref}
\usepackage[utf8]{inputenc}
\usepackage{layouts}
\usepackage{multirow}
\usepackage{multicol}
\usepackage{optidef}
\usepackage{physics}
\usepackage{setspace}
\usepackage{diagbox}
\usepackage[labelformat=simple]{subcaption}
\usepackage{xcolor}

% configure imports
\DeclareMathOperator*{\argmin}{arg\,min}
\newcommand{\todo}[1]{\textcolor{red}{TODO: #1}}
\definecolor{darkgreen}{RGB}{46, 184, 46}
\newcommand{\half}{\frac{1}{2}}
\newcommand{\R}{\mathbb{R}}
\captionsetup[subfigure]{labelformat=empty, skip=0pt}
\captionsetup{labelsep=period}
\restylefloat{table}
\renewcommand\thesubfigure{(\alph{subfigure})}
\addto\captionsenglish{%
  \renewcommand{\figurename}{FIG.}%
  \renewcommand{\tablename}{TABLE}%
  \renewcommand{\appendixname}{APPENDIX}%
}

\title{\vspace{-2em}NLOptics}
\date{}

\begin{document}

\maketitle

\vspace{-4em}
\begin{multicols}{3}

%% \textbf{Maxwell}
%% \begin{align}
%%   \vec{\nabla} \cdot \vec{E} &= \rho / \epsilon_0\\
%%   \vec{\nabla} \cdot \vec{B} &= 0\\
%%   \vec{\nabla} \cross \vec{E} &= - \partial_t \vec{B}\\
%%   \vec{\nabla} \cross \vec{B} &= \mu_0 \vec{J} +
%%   \mu_0 \epsilon_0 \partial_t \vec{E}\\
%%   \vec{E} &= -\partial_t \vec{A} - \vec{\nabla} \phi\\
%%   \vec{B} &= \vec{\nabla} \cross \vec{A}\\
%%   \vec{D} &= \epsilon_0 \vec{E} + \vec{P}\\
%%   \vec{H} &= \vec{B}/\mu_0 - \vec{M}\\
%%   \rho &= \rho_f + \rho_b\\
%%   \rho_b &= -\vec{\nabla} \cdot \vec{P}\\
%%   \vec{J} &= \vec{J}_{f} + \vec{J}_{b}\\
%%   \vec{J}_b &= \vec{\nabla} \cross \vec{M} + \partial_t \vec{P}
%% \end{align}
%% \begin{align}
%%   \phi = \frac{1}{4\pi\epsilon_0} \int &\frac{
%%     \rho(\vec{x}^{\prime}, t - \frac{\lvert\vec{x} - \vec{x}^{\prime}\rvert}{c})}{
%%     \lvert \vec{x} - \vec{x}^{\prime} \rvert} d^3 x^{\prime}\\
%%   \vec{A} = \frac{\mu_0}{4\pi} \int &\frac{\vec{J}(\vec{x}^{\prime},
%%     t - \frac{\lvert \vec{x} - \vec{x}^{\prime} \rvert}{c})}
%%   {\lvert \vec{x} - \vec{x}^{\prime} \rvert} d^3 x^{\prime}\\
%%   \int_V (\vec{\nabla} \cdot \vec{F}) dV &= \int_{\partial V} \vec{F} \cdot \hat{n} d(\partial V)\\
%%   \int_{S} (\vec{\nabla} \cross \vec{F}) \cdot dS &= \int_{\partial S} \vec{F} \cdot d(\partial S)
%% \end{align}
%% \begin{align}
%%   \vec{S} &= \vec{E} \cross \vec{H}\\
%%   \langle \vec{S} \rangle &= \frac{1}{2}
%%   \textrm{Re}[\vec{E}^{*} \cross \vec{H}]\\
%%   (\nabla^2 + k^2) &G_{H} =
%%   -\delta(\vec{x} - \vec{x}^{\prime})\\
%%   G_H &= \frac{e^{ik\lvert \vec{x}
%%       - \vec{x}^{\prime} \rvert}}{4\pi \lvert \vec{x} - \vec{x}^{\prime} \rvert}\\
%%   \mu_0 \epsilon_0 \partial_t \phi_L &= -\vec{\nabla} \cdot \vec{A}_L
%% \end{align}
%% \begin{align}
%%   \begin{split}
%%     G_{L-} = &\textstyle{\sum\limits_{\ell, m}} {}\frac{4\pi}{2\ell + 1}
%%     \frac{r^{\ell}}{r^{\prime \ell + 1}}\\
%%   &Y^{*}_{\ell, m}(\theta^{\prime}, \phi^{\prime})
%%   Y_{\ell, m}(\theta, \phi)
%%   \end{split}\\
%%   \begin{split}
%%   G_{L+} = &{}\textstyle{\sum\limits_{\ell, m}}
%%   \frac{4\pi}{2\ell + 1} \frac{r^{\prime \ell}}{r^{\ell + 1}}\\
%%   &Y^{*}_{\ell, m}(\theta^{\prime}, \phi^{\prime})
%%   Y_{\ell, m}(\theta, \phi)
%%   \end{split}\\
%%   \vec{n}_{12} &\times {\vec{E}_2 - \vec{E}_1} = 0\\
%%   \vec{n}_{12} &\times {\vec{H}_2 - \vec{H}_1} = \vec{J}_s\\
%%   \vec{n}_{12} &\cdot (\vec{D}_2 - \vec{D}_1) = \sigma_s\\
%%   \vec{n}_{12} &\cdot (\vec{B}_2 - \vec{B}_1) = 0\\
%%   c &= \epsilon^{-1/2}_0\mu^{-1/2}_0\\
%%   v &= \epsilon^{-1/2}\mu^{-1/2}\\
%%   \begin{split}
%%     n &= c/v\\
%%     &= \sqrt{\frac{\epsilon\mu}{\epsilon_0\mu_0}}
%%   \end{split}\\
%%   c &= \lambda f = \omega/k_0\\
%%   \begin{split}
%%     v_g &{}= \partial_k \omega\\
%%     &= c (n - \lambda \partial_{\lambda} n)^{-1}
%%   \end{split}\\
  %% \end{align}

\textbf{NL}
\begin{align}
  \bar{\bar{\kappa}}_{ij} &= \bar{\bar{\epsilon}}_{ij}^{-1} = \epsilon_0^{-1} (n_{ij}^{-2} \delta_{ij} + r_{ijk} E_k)\\ % Yariv eq 9.1-3
  \bar{P}^{(i)} &= \epsilon_0 \bar{\bar{\chi}}^{(i)} \bar{E}^{i}\\
  d_{ijk} &= \frac{1}{2} \chi_{ijk}\\ %% Boyd
  n_{e}(\theta) &= (n_e^{-2}\sin^2\theta + n_0^{-2}\cos^2\theta)^{-1/2}\\ %% Boyd eq 2.3.8
  \begin{split}
    \tan\rho &{}= \frac{\lvert\hat{S} \cross \hat{k}\rvert}{\hat{S} \cdot \hat{k}}\\
    &= \frac{\lvert \bar{E} \cross \bar{D} \rvert}{\bar{E} \cdot \bar{D}}\\
    &= \frac{1}{2}n_e(\theta)^2\sin2\theta(n_e^{-2} - n_o^{-2} )\\
    &(\textrm{Type I - Neg. Uniaxial})
  \end{split}
\end{align}
For walk-off angle, choose outgoing wave $D_k$ in kDB to lie along axis chosen for phase matching.
Transform out of kDB. Use $\bar{E} = \bar{\bar{\kappa}} \bar{D}$ to find outgoing E field. Use
formula for $\tan\rho$.

\textbf{kDB}
\begin{align}
  \bar{\bar{T}} &=
  \begin{pmatrix}
    \cos\theta\cos\phi & -\sin\phi & \sin\theta\cos\phi\\
    \cos\theta\sin\phi & \cos\phi & \sin\theta\sin\phi\\
    -\sin\theta & 0 & \cos\theta
  \end{pmatrix}\\
  &= R_z(\phi) R_y(\theta)
\end{align}
\begin{align}
  \begin{split}
    &\begin{pmatrix} \hat{e}_1 = e & \hat{e}_2 = o & \hat{e}_3=\hat{k}\end{pmatrix}^T\\
    &= \bar{\bar{T}}
    \begin{pmatrix} \hat{x} & \hat{y} & \hat{z}\end{pmatrix}^T
  \end{split}
\end{align}
\begin{align}
  \bar{E}_k &= \bar{\bar{\kappa}}_k \bar{D}_k + \bar{\bar{\chi}}_k \bar{B}_k\\
  \bar{H}_k &= \bar{\bar{\nu}}_k \bar{B}_k + \bar{\bar{\gamma}}_k \bar{D}_k
\end{align}
\begin{align}
  \begin{split}
  &\begin{pmatrix}
    \kappa_{11} & \kappa_{12}\\
    \kappa_{21} & \kappa_{22}
  \end{pmatrix}
  {}\begin{pmatrix}
    D_1\\
    D_2
  \end{pmatrix} =\\
  -&\begin{pmatrix}
    \chi_{11} & \chi_{12} - u\\
    \chi_{21} + u & \chi_{22}
  \end{pmatrix}
  \begin{pmatrix}
    B_1\\
    B_2
  \end{pmatrix}
  \end{split}\\
  \begin{split}
  &\begin{pmatrix}
    \nu_{11} & \nu_{12}\\
    \nu_{21} & \nu_{22}
  \end{pmatrix}
  {}\begin{pmatrix}
    B_1\\
    B_2
  \end{pmatrix} =\\
  -&\begin{pmatrix}
    \gamma_{11} & \gamma_{12} + u\\
    \gamma_{21} - u & \gamma_{22}
  \end{pmatrix}
  \begin{pmatrix}
      D_1\\
      D_2
  \end{pmatrix}
  \end{split}\\
  u &= \omega / k
\end{align}
%% Derive by considering $\bar{\nabla} \cross \bar{E}$, $\bar{\nabla} \cross \bar{H}$,
%% $\bar{\nabla} \cdot \bar{B}$, $\bar{\nabla} \cdot \bar{D}$ source-free, then FT.

%% \textbf{NL Poynting}
%% \begin{align}
%%   (\nabla \cross E &= -i \omega \mu_0 H) \cdot H^{*}\\
%%   (\nabla \cross H &= J + i \omega \bar{\bar{\epsilon}} E) \cdot E^{*}\\
%%   \begin{split}
%%     \nabla \cdot (E \cross H^{*}) &= (\nabla \cross E) \cdot H^*\\
%%     &- (\nabla \cross H^{*}) \cdot E
%%   \end{split}\\
%%   \begin{split}
%%     \nabla \cdot (E \cross H^*) - i \omega &(E \bar{\bar{\epsilon}} E^* - \mu_0 H H^*)\\
%%     &+ E \cdot J^* = 0
%%   \end{split}
%% \end{align}
%% For lossless media, time-averaged power transferred to medium is zero.
%% \begin{align}
%%   \textrm{Re}[i\omega E \bar{\bar{\epsilon}} E^*] &= 0
%% \end{align}
%% Consider $E = E \hat{x}$ and $E = E_x \hat{x} + E_y \hat{y}$ to derive properties
%% of $\bar{\bar{\epsilon}}_{ij}$.

\textbf{NL WE}
\begin{align}
  \begin{split}
    i2k\partial_zE + i\omega\mu_0\sigma E &+ i2\omega\mu_0\epsilon \partial_t E\\
    = \mu_0\omega^2(\hat{e} \cdot \hat{p})&e^{i(k-k_p)z}P_{\textrm{NL}}
  \end{split}\\
  k \partial_z E &\gg \partial_z^2 E\\
  \omega \partial_t E &\gg \partial_t^2 E\\
  \omega^2 P_{\textrm{NL}} \gg \omega \partial_t P_{\textrm{NL}} &\gg \partial_t^2 P_{\textrm{NL}}\\
  \vec{\nabla} \cross \vec{\nabla} \cross \vec{E} = \vec{\nabla} \cross &(-\mu_0 \partial_t \vec{H})
\end{align}

%% \textbf{Jones}
%% \begin{align}
%%   V_{RC} &= 2^{-1/2} \begin{pmatrix}1\\e^{i3\pi/2}\end{pmatrix}\\
%%   V_{LC} &= 2^{-1/2} \begin{pmatrix}1\\e^{i\pi/2}\end{pmatrix}\\
%%   J_{VWP} &= \begin{pmatrix} 1 & 0\\0 & e^{i \phi}\end{pmatrix}\\
%%   \begin{split}
%%     \vec{S} {}= &(\abs{E_x}^2 - \abs{E_y}^2) \hat{x}\\
%%     &+ 2\textrm{Re}[E_x^{*}E_y] \hat{y}\\
%%     &+ 2\textrm{Im}[E_x^{*}E_y] \hat{z}
%%   \end{split}\\
%%   M_{\theta} &= \begin{pmatrix}
%%     \cos\theta & -\sin\theta\\
%%     \sin\theta & \cos\theta
%%   \end{pmatrix}\\
%% \end{align}

\textbf{QPM}
\begin{align}
  \frac{w}{\Lambda} &= \frac{1}{2m}\\
  k_{m} &= \frac{2\pi m}{\Lambda} = \Delta k\\
  \begin{split}
    C_m &= \frac{2}{m\pi}\sin(\frac{m\pi w}{\Lambda}) \quad (m \neq 0)\\
    &= \frac{2w}{\Lambda} - 1 \quad (m = 0)
  \end{split}
\end{align}
%% TODO: remove after semester

\textbf{Gaussian Beam}
\begin{align}
  \begin{split}
    u_{00} &{}= 2^{1/2}\pi^{-1/2}w^{-1}e^{i\phi}\\
    &e^{-(x^2+y^2)/w^2}e^{-i(k/2R)(x^2+y^2)}
  \end{split}\\
  w^2 &= w_0^2[1+{(z/z_R)}^2]\\
  R &= (z^2 + z_R^2) / z\\
  \textrm{tan}\phi &= z/z_R\\
  z_R &= k w_0^2 / 2\\
  2 \omega_0 &\approx \frac{2\lambda}{\pi n\sin\theta}
\end{align}
%% TODO: remove after semester

\textbf{Misc Technical}
\begin{align}
  2\cos(x) &= e^{ix} + e^{-ix}\\
  i2\sin(x) &= e^{ix} - e^{-ix}\\
  2\cosh(x) &= e^x + e^{-x}\\
  2\sinh(x) &= e^x - e^{-x}\\
  \partial_x \sinh(x) &= \cosh(x)\\
  \partial_x \cosh(x) &= \sinh(x)\\
  \epsilon_{ijk} \epsilon_{ilm} &= \delta_{jl} \delta_{km} - \delta_{jm} \delta_{kl}\\
  f(x) = &\sum_{n = -\infty}^{\infty} c_n e^{i2\pi n x / P}\\
  c_n = \frac{1}{P} \int_{P} &f(x) e^{-i2\pi n x / P}\\
  \int_{\mathbb{R}} &(\frac{\sin x}{x})^2dx = \pi
\end{align}

%% \textbf{Misc Non-technical}
%% FA-55.10\% Coffey. Parents no HS. Eagle Scout. King College. Dyslexia.
%% MLK, Ghandi, Parents, Robert E Lee. Hand-written notes. Meditation.
%% %% TODO: remove after semester

\textbf{Classical Polarization Model}
\begin{align}
P &= -Nex(t)\\
\mathcal{L} &= \partial_t^2 + \Gamma\partial_t + \omega_0^2\\
c(\omega) &= (\omega_0^2 - \omega^2 + i\Gamma\omega)^{-1}\\
\mathcal{L}x &= \frac{F}{m} - \alpha x^2\\
x &= x^{(1)} + x^{(2)} + x^{(3)} + \dots
\end{align}
Consider iterative solutions that include one higher order in $x^{(i)}$. Drop terms in the
EOM that are of order $i + 1$ and higher.

\end{multicols}

\pagebreak

\begin{multicols}{2}
  
  \textbf{NL WE GV}
  \begin{align}
      \partial_z E + v_g^{-1} \partial_t E &= -i \frac{\omega \mu_0 c}{2n} P^{NL} e^{i(k - k_p) z}\\
    D(z, t) &= \int \epsilon(\omega + \delta \omega) E(\omega + \delta \omega)
    e^{-i[kz - (\omega + \delta \omega)t]} d\delta\omega\\
    \begin{split}
      \partial_t^2 D = \int &[\omega^2 {}\epsilon(\omega) + 2 \omega \delta\omega \epsilon(\omega)
        + \omega^2 \epsilon^{\prime}(\omega) \delta\omega]\\
      &E(\omega + \delta\omega)
      e^{-i[kz - (\omega + \delta \omega)t]} d\delta\omega
    \end{split}\\
    v_g^{-1} &= \partial_\omega k\\
    2kv_g^{-1} &= 2\mu_0 \epsilon \omega + \mu_0 \omega^2 \epsilon^{\prime}(\omega)
  \end{align}
  Alternative Derivation,
  \begin{align}
    \begin{split}
      \partial_z E &= i[k(\omega) - k(\omega_0)] E\\
      &= i[\partial_{\omega} k (\omega - \omega_0) + \frac{1}{2}\partial_{\omega}^2 k (\omega - \omega_0)^2] E\\
      &= -\partial_{\omega} k \partial_t E - \frac{i}{2}\partial_{\omega}^2 k \partial^2_t E
    \end{split}
  \end{align}

  \textbf{NL Processes}
  
  Para(Amp/Osc)
  \begin{align}
    \partial_z E_1 + \alpha_1 E_1 &= -i \kappa_1 E_3 E_2^* e^{-i\Delta k z}\\
    E_1 &= E_{10} e^{\gamma z - i \Delta k z / 2}\\
    (\gamma - i \Delta k z / 2 + \alpha_1) E_1 &+ i \kappa_1 E_3 E_2^* = 0\\
    (\gamma + i \Delta k z / 2 + \alpha_2) E_2^* &-i \kappa_1 E_3^* E_1 = 0\\
    \begin{split}
      \kappa_i {}&= \frac{\mu_0 \epsilon_0 \omega_i^2 4 d_{\textrm{eff}} }{2 k_i}\\
      &= \frac{4\pi d_{\textrm{eff}}}{\lambda n}
    \end{split}\\
    \begin{split}
      \gamma_\pm = -\frac{1}{2}(\alpha_1 + \alpha_2) \; &{}\pm \; \{\frac{1}{4}(\alpha_1 + \alpha_2)^2\\
      - \alpha_1\alpha_2 -\frac{\Delta k^2}{4} &-i(\alpha_1-\alpha_2)\frac{\Delta k}{2} + \kappa_1\kappa_2\lvert E_3\rvert^2\}^{1/2}
    \end{split}\\
    g &= (\kappa_1 \kappa_2 {\lvert E_3 \rvert}^2 - \Delta k^2 / 4)^{1/2}\\
    \begin{split}
      E_1(z) = e^{-\alpha z - i\Delta k z / 2} {}&\{E_1(0) \cosh gz \\
      - i[\frac{\Delta k}{2 g}E_1(0) &+ \frac{\kappa_1}{g} E_3 E_2^*(0)] \sinh gz\}
    \end{split}\\
    G_1(z) &= \frac{{\lvert E_1(z) \rvert}^2}{{\lvert E_1(0) \rvert}^2} - 1\\
    \cosh gz &= 1 + \frac{\ell_1 \ell_2}{2 - \ell_1 - \ell_2}\\
    \ell_i &= 1 - R_ie^{-\alpha_i z}
  \end{align}

  Four Wave Mixing
  \begin{align}
    \omega_4 + \omega_3 &= \omega_1 + \omega_2\\
    E_1, E_2 &\; \textrm{const.}\\
    \partial_z E_4 &= -i \frac{\kappa_4}{\kappa_c} E_1 E_2 E_3^*\\
    \kappa_c &= \frac{\omega_2 d}{n_i c}
  \end{align}

  $\chi^{(3)}$
  \begin{align}
    \partial_z E + v_g^{-1} \partial_t E &= -i \frac{k^{\prime\prime}}{2} \partial_t^2 E
    + i \kappa \lvert E \rvert^2 E\\
    \kappa &= \frac{3 \omega \chi^{(3)}}{8 c n}\\
    \tau &= t - v_g^{-1} z\\
    \xi &= z\\
    \partial_\xi E &= -i \frac{k^{\prime\prime}}{2} \partial_\tau^2 E + i \kappa \lvert E \rvert^2 E\\
    E(\xi, \tau) &= A_0 \sech(\tau/\tau_0) e^{i\kappa \lvert A_0 \rvert^2 \xi}\\
    \tau_0^{-2} &= -\kappa \lvert A_0 \rvert^2 / k^{\prime\prime}\\
    \chi_{xxxx}^{(3)} &= T_{xi} \chi^{(3)}_{ijkl} T_{jx} T_{kx} T_{lx}
  \end{align}

  \textbf{Quantum}
  
  General
  \begin{align}
    \ket*{\dot{\psi}} &= -iH/\hbar\ket{\psi}\\
    \dot{\rho} &= -i [H/\hbar, \rho]\\
    H_1/\hbar &= U_1^{\dagger}H_0/\hbar U_1 - i U_{1}^{\dagger}\dot{U}_1\\
    \Gamma &= \frac{2\pi}{\hbar} \lvert \bra{f} H^{\prime} \ket{i} \rvert^2 \rho(E_f)\\
    \langle P \rangle &= \frac{N}{V}\langle \mu \rangle
  \end{align}

    Two-Level Atom
  \begin{align}
    \lvert c_n^{(1)} \rvert^2 &= \lvert \Omega \rvert^2 (\frac{\sin \Delta t / 2}{\Delta / 2})^2\\
    \dot{c}_n &= -i \Delta c_n + i \Omega c_g\\
    \rho_{ng}^{(1)} &= (\omega_{ng} - \nu - i\Gamma_2)^{-1} \Omega e^{-i\nu t}(\rho_{gg}^{(0)} - \rho_{nn}^{(0)})\\
    \Omega &= \frac{\mu_{ng} E}{2 \hbar}; \quad \Delta = \omega_{ng} - \nu
  \end{align}

  Perturbation Theory
  \begin{align}
    \dot{c}_{n}^{(n)} = -i \sum_m &e^{-i\omega_{mn} t} H_{nm}^{\prime}(t)/\hbar \; c_{m}^{(n - 1)}(t)\\ %% HW6
    \begin{split}
      \dot{\rho}_{mn}^{(n)} = -i \Omega_{mn} \rho_{mn}^{(n)}&\\
      -i \sum_{l} (H^{\prime}_{ml}&/\hbar \; \rho_{ln}^{(n - 1)} - \rho_{ml}^{(n - 1)} H^{\prime}_{ln}/\hbar) %% HW6
    \end{split}\\
    \Omega_{mn} &= \omega_{mn} - i\Gamma_{mn}\\
    A_{nm}(t_{j + 1} - t_j) &= e^{-i\Omega_{nm}(t_{j+1} - t_j)}\\
    H^{\prime}_{nm}(t_j) &= -\frac{1}{2}\mu_{nm}Ee^{-i\omega t_j}
  \end{align}

  \begin{enumerate}
  \item Use $\rho^{(0)}_{nm}(t_0)$
  \item For ket line from $\ket*{n}$ to $\ket*{m}$ through $t_j$, use $H_{mn}^{\prime}(t_j)$; take $\dagger$ for bra line;
    take $\dagger$ if photon leaves system
  \item For ket--bra lines along $\ket*{n}\bra*{m}$ from $t_j$ to $t_{j + 1}$, use $\int_{t_0}^{t_{j+1}} dt_j A_{nm}(t_{j + 1} - t_j)$
  \item Use $(i\hbar)^{-1}$ for each integration
  \item Steady state $\lim_{t_0 \rightarrow -\infty}$
  \end{enumerate}
  
\end{multicols}

\begin{table}
  \begin{tabular}{c|c|c}
    &(+) uniaxial ($n_e > n_o$)& (-) uniaxial ($n_e < n_o$)\\
    \hline
    Type I & $n_3^o \omega_3 = n_1^e \omega_1 + n_2^e \omega_2$ & $n_3^e \omega_3 = n_1^o \omega_1 + n_2^o \omega_2$\\
    Type II & $n_3^o \omega_3 = n_1^o \omega_1 + n_2^e \omega_2$ & $n_3^e \omega_3 = n_1^e \omega_1 + n_2^o \omega_2$\\
    && $\omega_3 > \omega_2 > \omega_1$\\
  \end{tabular}
\end{table}


\begin{table}
  \begin{tabular}{c|c}
    index type & crystal structure\\
    \hline
    isotropic & cubic (432, $\bar{4}$3m, 23, m3, m3m)\\
    \hline
    uniaxial & trigonal (3, 32, 3m, $\bar{3}$, $\bar{3}$m)\\
    & tetragonal (4, $\bar{4}$, 422, 4mm, $\bar{4}$2m, 4/m, 4/mmm)\\
    & hexagonal (6, $\bar{6}$, 622, 6mm, $\bar{6}$m2, 6/m, 6/mmm)\\
    \hline
    biaxial & triclinic (1, $\bar{1}$)\\
    & monoclinic (2, m, 2/m)\\
    & orthorhombic (222, mm2, mmm)\\
  \end{tabular} %% Boyd p 46, 76
\end{table}


\end{document}
