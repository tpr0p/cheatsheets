\documentclass[12pt]{article}

% imports
\usepackage{appendix}
\usepackage{amsmath}
\usepackage{amssymb}
\usepackage[english]{babel}
\usepackage{bbold}
\usepackage[justification=centering]{caption}
\usepackage{float}
\usepackage[margin=1em]{geometry}
\usepackage{graphicx}
\usepackage{hyperref}
\usepackage[utf8]{inputenc}
\usepackage{layouts}
\usepackage{multirow}
\usepackage{multicol}
\usepackage{optidef}
\usepackage{physics}
\usepackage{setspace}
\usepackage{diagbox}
\usepackage[labelformat=simple]{subcaption}
\usepackage{xcolor}

% configure imports
\DeclareMathOperator*{\argmin}{arg\,min}
\newcommand{\todo}[1]{\textcolor{red}{TODO: #1}}
\definecolor{darkgreen}{RGB}{46, 184, 46}
\newcommand{\half}{\frac{1}{2}}
\newcommand{\R}{\mathbb{R}}
\captionsetup[subfigure]{labelformat=empty, skip=0pt}
\captionsetup{labelsep=period}
\restylefloat{table}
\renewcommand\thesubfigure{(\alph{subfigure})}
\addto\captionsenglish{%
  \renewcommand{\figurename}{FIG.}%
  \renewcommand{\tablename}{TABLE}%
  \renewcommand{\appendixname}{APPENDIX}%
}

\title{\vspace{-2em}NLOptics}
\date{}

\begin{document}

\maketitle

\vspace{-4em}
\begin{multicols}{3}

\textbf{Maxwell}
\begin{align}
  \vec{\nabla} \cdot \vec{E} &= \rho / \epsilon_0\\
  \vec{\nabla} \cdot \vec{B} &= 0\\
  \vec{\nabla} \cross \vec{E} &= - \partial_t \vec{B}\\
  \vec{\nabla} \cross \vec{B} &= \mu_0 \vec{J} +
  \mu_0 \epsilon_0 \partial_t \vec{E}\\
  \vec{E} &= -\partial_t \vec{A} - \vec{\nabla} \phi\\
  \vec{B} &= \vec{\nabla} \cross \vec{A}\\
  \vec{D} &= \epsilon_0 \vec{E} + \vec{P}\\
  \vec{H} &= \vec{B}/\mu_0 - \vec{M}\\
  \rho &= \rho_f + \rho_b\\
  \rho_b &= -\vec{\nabla} \cdot \vec{P}\\
  \vec{J} &= \vec{J}_{f} + \vec{J}_{b}\\
  \vec{J}_b &= \vec{\nabla} \cross \vec{M} + \partial_t \vec{P}\\
  \phi = \frac{1}{4\pi\epsilon_0} \int &\frac{
    \rho(\vec{x}^{\prime}, t - \frac{\lvert\vec{x} - \vec{x}^{\prime}\rvert}{c})}{
    \lvert \vec{x} - \vec{x}^{\prime} \rvert} d^3 x^{\prime}\\
  \vec{A} = \frac{\mu_0}{4\pi} \int &\frac{\vec{J}(\vec{x}^{\prime},
    t - \frac{\lvert \vec{x} - \vec{x}^{\prime} \rvert}{c})}
  {\lvert \vec{x} - \vec{x}^{\prime} \rvert} d^3 x^{\prime}\\
  \int_V (\vec{\nabla} \cdot \vec{F}) dV &= \int_{\partial V} \vec{F} \cdot \hat{n} d(\partial V)\\
  \int_{S} (\vec{\nabla} \cross \vec{F}) \cdot dS &= \int_{\partial S} \vec{F} \cdot d(\partial S)\\
  \vec{S} &= \vec{E} \cross \vec{H}\\
  \langle \vec{S} \rangle &= \frac{1}{2}
  \textrm{Re}[\vec{E}^{*} \cross \vec{H}]\\
  (\nabla^2 + k^2) &G_{H} =
  -\delta(\vec{x} - \vec{x}^{\prime})\\
  G_H &= \frac{e^{ik\lvert \vec{x}
      - \vec{x}^{\prime} \rvert}}{4\pi \lvert \vec{x} - \vec{x}^{\prime} \rvert}\\
  \mu_0 \epsilon_0 \partial_t \phi_L &= -\vec{\nabla} \cdot \vec{A}_L
\end{align}
\begin{align}
  \begin{split}
    G_{L-} = &\textstyle{\sum\limits_{\ell, m}} {}\frac{4\pi}{2\ell + 1}
    \frac{r^{\ell}}{r^{\prime \ell + 1}}\\
  &Y^{*}_{\ell, m}(\theta^{\prime}, \phi^{\prime})
  Y_{\ell, m}(\theta, \phi)
  \end{split}\\
  \begin{split}
  G_{L+} = &{}\textstyle{\sum\limits_{\ell, m}}
  \frac{4\pi}{2\ell + 1} \frac{r^{\prime \ell}}{r^{\ell + 1}}\\
  &Y^{*}_{\ell, m}(\theta^{\prime}, \phi^{\prime})
  Y_{\ell, m}(\theta, \phi)
  \end{split}\\
  \vec{n}_{12} &\times {\vec{E}_2 - \vec{E}_1} = 0\\
  \vec{n}_{12} &\times {\vec{H}_2 - \vec{H}_1} = \vec{J}_s\\
  \vec{n}_{12} &\cdot (\vec{D}_2 - \vec{D}_1) = \sigma_s\\
  \vec{n}_{12} &\cdot (\vec{B}_2 - \vec{B}_1) = 0\\
  c &= \epsilon^{-1/2}_0\mu^{-1/2}_0\\
  v &= \epsilon^{-1/2}\mu^{-1/2}\\
  n &= c/v\\
  c &= \lambda f = \omega/k_0
\end{align}

\textbf{EO}
\begin{align}
  \bar{\bar{\kappa}}_{ij} &= \bar{\bar{\epsilon}}_{ij}^{-1} = \epsilon_0^{-1} (n_{ij}^{-2} \delta_{ij} + r_{ijk} E_k)\\ % Yariv eq 9.1-3
  \bar{P}^{(i)} &= \epsilon_0 \bar{\bar{\chi}}^{(i)} \bar{E}^{i}\\
\end{align}

\textbf{kDB}
\begin{align}
  \bar{\bar{T}} &=
  \begin{pmatrix}
    \sin\phi & -\cos\phi & 0\\
    \cos\theta\cos\phi & \cos\theta\sin\phi & -\sin\theta\\
    \sin\theta\cos\phi & \sin\theta\sin\phi & \cos\theta
  \end{pmatrix} % Kong eq 3.3.34
\end{align}
\begin{align}
  \begin{pmatrix} \hat{e}_1 & \hat{e}_2 & \hat{e}_3=\hat{k}\end{pmatrix}^T &= \bar{\bar{T}}
  \begin{pmatrix} \hat{x} & \hat{y} & \hat{z}\end{pmatrix}^T
\end{align}
\begin{align}
  \bar{E}_k &= \bar{\bar{\kappa}}_k \bar{D}_k + \bar{\bar{\chi}}_k \bar{B}_k\\
  \bar{H}_k &= \bar{\bar{\nu}}_k \bar{B}_k + \bar{\bar{\gamma}}_k \bar{D}_k
\end{align}
\begin{align}
  \begin{split}
  &\begin{pmatrix}
    \kappa_{11} & \kappa_{12}\\
    \kappa_{21} & \kappa_{22}
  \end{pmatrix}
  {}\begin{pmatrix}
    D_1\\
    D_2
  \end{pmatrix} =\\
  -&\begin{pmatrix}
    \chi_{11} & \chi_{12} - u\\
    \chi_{21} + u & \chi_{22}
  \end{pmatrix}
  \begin{pmatrix}
    B_1\\
    B_2
  \end{pmatrix}
  \end{split}\\
  \begin{split}
  &\begin{pmatrix}
    \nu_{11} & \nu_{12}\\
    \nu_{21} & \nu_{22}
  \end{pmatrix}
  {}\begin{pmatrix}
    B_1\\
    B_2
  \end{pmatrix} =\\
  -&\begin{pmatrix}
    \gamma_{11} & \gamma_{12} + u\\
    \gamma_{21} - u & \gamma_{22}
  \end{pmatrix}
  \begin{pmatrix}
      D_1\\
      D_2
  \end{pmatrix}
  \end{split}
\end{align}
Derive by considering $\bar{\nabla} \cross \bar{E}$, $\bar{\nabla} \cross \bar{H}$,
$\bar{\nabla} \cdot \bar{B}$, $\bar{\nabla} \cdot \bar{D}$ source-free, then FT.

\textbf{NL Poynting}
\begin{align}
  (\nabla \cross E &= -i \omega \mu_0 H) \cdot H^{*}\\
  (\nabla \cross H &= J + i \omega \bar{\bar{\epsilon}} E) \cdot E^{*}\\
  \begin{split}
    \nabla \cdot (E \cross H^{*}) &= (\nabla \cross E) \cdot H^*\\
    &- (\nabla \cross H^{*}) \cdot E
  \end{split}\\
  \begin{split}
    \nabla \cdot (E \cross H^*) - i \omega &(E \bar{\bar{\epsilon}} E^* - \mu_0 H H^*)\\
    &+ E \cdot J^* = 0
  \end{split}
\end{align}
For lossless media, time-averaged power transferred to medium is zero.
\begin{align}
  \textrm{Re}[i\omega E \bar{\bar{\epsilon}} E^*] &= 0
\end{align}
Consider $E = E \hat{x}$ and $E = E_x \hat{x} + E_y \hat{y}$ to derive properties
of $\bar{\bar{\epsilon}}_{ij}$.

\textbf{NL WE}
\begin{align}
  \begin{split}
    i2k\partial_zE + i\omega\mu_0\sigma E &+ i2\omega\mu_0\epsilon \partial_t E\\
    = \mu_0\omega^2(\hat{e} \cdot \hat{p})&e^{i(k-k_p)z}P_{\textrm{NL}}
  \end{split}\\
  k \partial_z E &\gg \partial_z^2 E\\
  \omega \partial_t E &\gg \partial_t^2 E\\
  \omega^2 P_{\textrm{NL}} \gg \omega \partial_t P_{\textrm{NL}} &\gg \partial_t^2 P_{\textrm{NL}}\\
  \vec{\nabla} \cross \vec{\nabla} \cross \vec{E} = \vec{\nabla} \cross &(-\mu_0 \partial_t \vec{H})
\end{align}

\textbf{Jones}
\begin{align}
  V_{RC} &= 2^{-1/2} \begin{pmatrix}1\\e^{i3\pi/2}\end{pmatrix}\\
  V_{LC} &= 2^{-1/2} \begin{pmatrix}1\\e^{i\pi/2}\end{pmatrix}\\
  J_{VWP} &= \begin{pmatrix} 1 & 0\\0 & e^{i \phi}\end{pmatrix}\\
  \begin{split}
    \vec{S} {}= &(\abs{E_x}^2 - \abs{E_y}^2) \hat{x}\\
    &+ 2\textrm{Re}[E_x^{*}E_y] \hat{y}\\
    &+ 2\textrm{Im}[E_x^{*}E_y] \hat{z}
  \end{split}\\
  M_{\theta} &= \begin{pmatrix}
    \cos\theta & -\sin\theta\\
    \sin\theta & \cos\theta
  \end{pmatrix}\\
\end{align}

\textbf{Misc NL}
\begin{align}
  P(t) &= N (-e) x(t)\\
  \bar{P}^{(i)} &= \epsilon_0 \bar{\bar{\chi}}^{(i)} \bar{E}^{i}\\
\end{align}

\textbf{Misc Technical}
\begin{align}
  2\cos(\theta) &= (e^{i\theta} + e^{-i\theta})\\
  i2\sin(\theta) &= (e^{i\theta} - e^{-i\theta})
\end{align}

\textbf{Misc Non-technical}
FA-55.10\% Coffey. Parents no HS. Eagle Scout. King College. Dyslexia.
MLK, Ghandi, Parents, Robert E Lee. Hand-written notes. Meditation.

%% TODO:
%% NL poynting + derivation

\end{multicols}

\end{document}
