\documentclass[12pt]{article}

% imports
\usepackage{appendix}
\usepackage{amsmath}
\usepackage{amssymb}
\usepackage[english]{babel}
\usepackage{bbold}
\usepackage[justification=centering]{caption}
\usepackage{float}
\usepackage[margin=1em]{geometry}
\usepackage{graphicx}
\usepackage{hyperref}
\usepackage[utf8]{inputenc}
\usepackage{layouts}
\usepackage{multirow}
\usepackage{multicol}
\usepackage{optidef}
\usepackage{physics}
\usepackage{setspace}
\usepackage{diagbox}
\usepackage[labelformat=simple]{subcaption}
\usepackage{xcolor}

% configure imports
\DeclareMathOperator*{\argmin}{arg\,min}
\newcommand{\todo}[1]{\textcolor{red}{TODO: #1}}
\definecolor{darkgreen}{RGB}{46, 184, 46}
\newcommand{\half}{\frac{1}{2}}
\newcommand{\R}{\mathbb{R}}
\captionsetup[subfigure]{labelformat=empty, skip=0pt}
\captionsetup{labelsep=period}
\restylefloat{table}
\renewcommand\thesubfigure{(\alph{subfigure})}
\addto\captionsenglish{%
  \renewcommand{\figurename}{FIG.}%
  \renewcommand{\tablename}{TABLE}%
  \renewcommand{\appendixname}{APPENDIX}%
}

\title{\vspace{-2em}Geometry}
\date{}

\begin{document}

\maketitle

\vspace{-4em}
\begin{multicols}{2}

  \begin{center}
    \textbf{Linear Algebra}
  \end{center}
  
  \noindent
  \textbf{Def. Vector Space:} A vector space $(V, +, \cdot)$ on a scalar field $\mathbb{F}$ is,
  \begin{enumerate}
  \item a set $V$,
  \item $+ : V \times V \rightarrow V$,
  \item $\cdot : \mathbb{F} \times V$ ,
  \end{enumerate}
  satisfying,
  \begin{enumerate}
  \item Commutivity $(+)$,
  \item Associativity $(+)$,
  \item Neutral $(+)$,
  \item Inverse $(+)$,
  \item Associativity - scalar $(\cdot)$,
  \item Distributivity - scalar $(\cdot)$,
  \item Distributivity - vector $(\cdot)$,
  \item Unity $(\cdot)$.
  \end{enumerate}
  
  \noindent
  \textbf{Def. Linear Map:} For vector spaces $V$, $W$ a map $\phi: V \rightarrow W$ is linear ($\tilde{\rightarrow}$) if
  for all $u, v \in V$ and $\lambda \in \mathbb{F}$,
  \begin{enumerate}
  \item $\phi(u +_V v) = \phi(u) +_W \phi(v)$,
  \item $\phi(\lambda \cdot_V v) = \lambda \cdot_W \phi(v)$.
  \end{enumerate}
  
  \noindent
  \textbf{Def. Dual Space:} The dual vector space to $V$ is $V^{*} = \{\phi: V \tilde{\rightarrow} \mathbb{F}\}$.
  
  \noindent
  \textbf{Def. Tensor:} A ($r$, $s$) tensor is a multi-linear
  map $T: V^{*} \times \dots \times V^{*} \times V \times \dots \times V \rightarrow \mathbb{F}$.

  \begin{center}
    \textbf{Topology}
  \end{center}
  
  \noindent
  \textbf{Def. Topology:} A topology $O \subset P(M)$ satisfies,
  \begin{enumerate}
  \item $\emptyset \in O$, $M \in O$,
  \item $U \in O$, $V \in O$ $\implies$ $U \cap V \in O$,
  \item $U_{\alpha} \in O \implies \cup_{\alpha} U_{\alpha} \in O$.
  \end{enumerate}
  
  \noindent
  \textbf{Def. Continuity:} $(M, O_M), (N, O_N)$ top. spaces.
  A map $f: M \rightarrow N$ is continuous w.r.t. to the topologies if for all $U \in O_N$,
  $\textrm{preim}_f(U) \in O_M$.
  
  \noindent
  \textbf{Def. Topological Manifold:} $(M, O)$ called a top. manifold if for all $p \in M$, there exists
  $U$ with $p \in U \in O$ and there exists $x: U \rightarrow x(U) \subset \mathbb{R}^d$ with
  \begin{enumerate}
  \item $x$ continuous,
  \item $x^{-1}$ exists,
  \item $x^{-1}$ continuous.
  \end{enumerate}
  $A = {(U, x)}$ an atlas if $M = \cup_{\alpha} U_{\alpha}$.
  
  \noindent
  \textbf{Def. $*$ Compatability:} Two charts $(U, x), (V, y)$ of a top. manifold are called $*$-compatible if either
  $U \cap V = \emptyset$ or $U \cap V \neq \emptyset$ and $y \circ x^{-1}$ and $x \circ y^{-1}$ are $*$-differentiable.
  
  \noindent
  \textbf{Def. $*$ Manifold:} A $*$-manifold is $(M, O, A)$ where
  $(M, O)$ is a top. manifold and any two charts in $A$ are $*$-compatible.
  The manifold is called smooth if $* = C^{\infty}$.

  \noindent
  \textbf{Thm.}
  $C^{\infty}(M) = \{f : M \rightarrow \mathbb{R} | f \ \textrm{is} \ C^{\infty}\}$ is a ring.

  \noindent
  \textbf{Def. Velocity:} $(M, O, A)$ is a smooth manifold. $\gamma : \mathbb{R} \rightarrow M$
  is a curve at least $C^{1}$ with $\gamma(\lambda_0) = p$. The velocity of $\gamma$ at $p$ is the linear map
  $v_{\gamma, p} : C^{\infty}(M) \rightarrow \mathbb{R}$ given by $v_{\gamma, p}(f) = \partial_1 (f \circ \gamma)\lvert_{\lambda_0}$.

  \noindent
  \textbf{Def. Tangent Space:} $T_{p}M = \{v_{\gamma, p} | \gamma \ C^{\infty}\}$

  \noindent
  \textbf{Def. Cotangent Space:} $T^{*}_{p}M = \{\phi: T_pM \tilde{\rightarrow} \mathbb{R}\}$

  %% TODO: I don't understand this construction fully, what is the def. of the action dx^i
  \noindent
  \textbf{Thm. Tangent Space Bases:} For a chart $(U, x)$, $\partial x_i \lvert_p$ constitutes a basis for $T_{p}U$
  and $d x^i$ constitutes a basis for $T^{*}_{p}U$ where $\partial x_i \lvert_p f = \partial_i (f \circ x^{-1})\lvert_{x(p)}$,
  for $f \in C^{\infty}(M)$.

  \noindent
  \textbf{Def. Bundle:} A bundle is $(E, \pi, M)$ where $E$ and $M$
  are smooth mainfolds and $\pi: E \rightarrow M$ is a smooth, surjective map.

  \noindent
  \textbf{Def. Fiber:} Given a bundle $(E, \pi, M)$, a fiber over $p \in M$
  is $\textrm{preim}_\pi({p})$.

  \noindent
  \textbf{Def. Section:} Given a bundle $(E, \pi, M)$, a section $\sigma$ is
  a map $\sigma: M \rightarrow E$ satisfying $\pi \circ \sigma = \textrm{id}_M$.

  \noindent
  \textbf{Thm. Tangent Bundle:} The "tangent bundle"
  $TM = \dot{\cup}_{p \in M} T_pM$ is a smooth manifold
  when equipped with,
  \begin{align}
  O_{TM} &= \{\textrm{preim}_{\pi}(U) | U \in O\},\\
  A_{TM} &= \{(TU, \xi_x), | (U, x) \in A\},
  \end{align}
  where $\xi_X: TU \rightarrow \mathbb{R}^{2d}$,
  \begin{align}
  \xi_x(X) &= (\dots, (x^i \circ \pi)(X), \dots, dx^i(X), \dots), \ \textrm{and}\\
  \pi(X) &= p \ \textrm{s.t.} \ X \in T_pM.
  \end{align}
  This structure allows the creation of a bundle $(TM, \pi, M)$.
  A similar structure can make $T^*M = \dot{\cup}_{p \in M} T_p^*M$
  a smooth manifold.

  \noindent
  \textbf{Def. Vector Field:}
  A vector field is a smooth section of the tangent bundle.

  %% TODO:  I don't understand this construction fully.
  \noindent
  \textbf{Thm. Tangent Bundle Module:}
  $\Gamma(TM) = \{\sigma : M \rightarrow TM\}$ is a vector space over
  $C^{\infty}(M)$ with the following operations for $\sigma, \nu \in \Gamma(TM)$
  and $f, g \in C^{\infty}(M)$,
  \begin{itemize}
  \item $(\sigma \oplus \nu)(f) = \sigma(f) + \nu(f)$,
  \item $(g \odot \sigma)(f) = g \cdot \sigma(f)$
  \end{itemize}

  \noindent
  \textbf{Def. Tensor Field:}
  A (r, s) tensor field is a multi-linear map $T: \Gamma(T^*M) \times \dots \times \Gamma(T^*M) \times \Gamma(TM) \times \dots \times \Gamma(TM) \tilde{\rightarrow} C^{\infty}(M)$. It is understood
  that a $(0, 0)$ tensor field takes $T: C^{\infty}(M) \tilde{\rightarrow} C^{\infty}(M)$.

  \noindent
  \textbf{Def. Connection:} A connection on a smooth manifold is a map that takes a vector field
  and a $(p, q)$ tensor field $T$ and returns a $(p, q)$ tensor field, satisfying, for all $f \in C^{\infty}(M)$,
  $X, Y, Z \in \Gamma(TM)$, and $w \in \Gamma(T^*M)$,
  \begin{itemize}
  \item $\nabla_X f = X f$ 
  \item $\nabla_X (T + S) = \nabla_X T + \nabla_X S$
  \item $\nabla_X (T(w, Y)) = (\nabla_X T)(w, Y) + T(\nabla_X w, Y) + T(w, \nabla_X Y)$
  \item $\nabla_{f X + Z} T = f \nabla_X T + \nabla_Z T$
  \end{itemize}

  \noindent
  \textbf{Rmk. Connection Coefficient Functions:}
  \begin{align}
  \nabla_{\partial x_k} \partial x_j &= {\Gamma_{(x)}}^i_{jk} \partial x_i\\
  \nabla_{\partial x_k} d x^i &= -{\Gamma_{(x)}}^i_{jk} dx^j\\
  \begin{split}
    {\Gamma_{(y)}}^i_{jk} &= (\partial x_q y^i) [\partial y_k (\partial y_j x^q)]\\
    & \ + (\partial x_q y^i) (\partial y_k x^p) (\partial y_j x^s) {\Gamma_{(x)}}^q_{sp}
  \end{split}
  \end{align}
  
  \noindent
  \textbf{Rmk.}
  \begin{align}
    A^{[ab]} &= \frac{1}{2}(A^{ab} - A^{ba})\\
    A^{(ab)} &= \frac{1}{2}(A^{ab} + A^{ba})
  \end{align}

  \noindent
  \textbf{Def. Torsion:}
  The torision of a connection is the (1, 2) tensor field
  \begin{align}
    T(w, X, Y) = w(\nabla_X Y - \nabla_Y X - [X, Y])
  \end{align}
  A manifold with connection is torsion free if the torsion tensor vanishes everywhere on the manifold.
  It can be shown that the connection coefficient functions are symmetric in the lower two indices
  if the connection is torsion free.

  \noindent
  \textbf{Def. Riemann Curvature:}
  The Riemann Curvature of a connection is the (1, 3) tensor field
  \begin{align}
    T(w, Z, X, Y) = w(\nabla_X \nabla_Y Z - \nabla_Y \nabla_X Z - \nabla_{[X, Y]} Z)
  \end{align}

  \noindent
  \textbf{Def. Parallel Transport:}
  A vector field $X$ is parallely transported along a smooth curve $\gamma(\lambda)$ if,
  \begin{align}
    \nabla_{v_{\gamma, \gamma(\lambda)}} X \lvert_{\gamma(\lambda)} &= 0,  \ \textrm{for all} \ \lambda.
  \end{align}
  A vector field is parallel to another vector $\mu X$ with $\mu: \mathbb{R} \rightarrow \mathbb{R}$ if,
  \begin{align}
    \nabla_{v_{\gamma, \gamma(\lambda)}} X \lvert_{\gamma(\lambda)} &= \mu \lvert_{\lambda} X \lvert_{\gamma(\lambda)},
    \ \textrm{for all} \ \lambda.
  \end{align}

  \noindent
  \textbf{Def. Autoparallel Transport:}
  A vector field is autoparallely transported if,
  \begin{align}
    \nabla_{v_{\gamma, \gamma(\lambda)}} v_{\gamma, \gamma(\lambda)} &= 0, \ \textrm{for all} \ \lambda.
  \end{align}
  A vector field is autoparallel to another $\mu v_{\gamma}$ if,
  \begin{align}
    \nabla_{v_{\gamma, \gamma(\lambda)}} v_{\gamma, \gamma(\lambda)} &= \mu \lvert_{\lambda} v_{\gamma, \gamma(\lambda)},
    \ \textrm{for all} \ \lambda.
  \end{align}

  \noindent
  \textbf{Def. Metric:}
  A metric $g$ is a $(0, 2)$ tensor field satisfying,
  \begin{itemize}
  \item symmetric: $g(X, Y) = g(Y, X)$,
  \item non-degenerate: for every $X \in \Gamma(TM),\ X \neq 0$, there exists $Y \in \Gamma(TM)$ s.t. $g(X, Y) \neq 0$,
  \end{itemize}
  A Riemannian metric is positive definite.
  The inverse metric $g^{-1}$ is a symmetric, non-degenerate $(2, 0)$ tensor satisfying $g(w, \sigma) = w b^{-1}(\sigma)$
  where $b(x) = g(x, \cdot)$ is a $C^{\infty}$ isomorphism which exists and is invertible by the non-degeneracy condition
  on $g$. Note that ${g^{-1}}^{am}g_{mb} = \delta^a_b$.

  \noindent
  \textbf{Def. Length:}
  The speed of a curve $\gamma: (a, b) \subset \mathbb{R} \rightarrow M$ is,
  \begin{align}
    s(\lambda) &= [g(v_{\gamma}, v_{\gamma})]^{1/2} \lvert_{\gamma(\lambda)}.
  \end{align}
  The length of a curve is,
  \begin{align}
    L[\gamma] &= \int_a^b d\lambda \ s(\lambda).
  \end{align}
  It can be shown that this length is invariant under reparameterizations of $\gamma$.

  \noindent
  \textbf{Def. Geodesic:} A curve $\gamma$ is called a geodesic if it is stationary (extremal)
  w.r.t. $L[\gamma]$.

  \noindent
  \textbf{Thm. Geodesic Equation:} $\gamma$ is a geodesic iff it satisfies,
  \begin{align}
    \ddot{\gamma}^q &+ \Gamma^q_{ij}[\gamma(\lambda)] \dot{\gamma}^i \dot{\gamma}^j = 0,\\
    \Gamma^q_{ij} &= (g^{-1})^{qm} \frac{1}{2}(\partial_i g_{mj} + \partial_j g_{mi} - \partial_m g_{ij}),
  \end{align}
  where $\dot{\gamma}^i = \partial_1 (x \circ \gamma)^i$.

  \noindent
  \textbf{Def. Levi-Civita:}
  The $\Gamma^q_{ij}$ in the geodesic equation are the connection coefficient functions for the Levi-Civita connection,
  also called the Christoffel Symbols of the first kind.
  An equivalent way to define this connection is requiring $\nabla g = 0$ and $T_{\textrm{torsion}} = 0$.

  \noindent
  \textbf{Def. Riemann Christoffel Curvature:}
  \begin{align}
    R_{abcd} &= g_{am} R^m_{bcd},
  \end{align}
  where the Riemann curvature is obtained with the Levi-Civita connection.

  \noindent
  \textbf{Def. Ricci Curvature:}
  \begin{align}
    R_{ab} &= R^m_{amb}\\
    R &= (g^{-1})^{ab} R_{ab}
  \end{align}

  
\end{multicols}

\end{document}
