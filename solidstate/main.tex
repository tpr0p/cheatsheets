\documentclass[12pt]{article}

% imports
\usepackage{appendix}
\usepackage{amsmath}
\usepackage{amssymb}
\usepackage[english]{babel}
\usepackage{bbold}
\usepackage[justification=centering]{caption}
\usepackage{float}
\usepackage{graphicx}
\usepackage{hyperref}
\usepackage[utf8]{inputenc}
\usepackage{layouts}
\usepackage{multirow}
\usepackage{optidef}
\usepackage{physics}
\usepackage{setspace}
\usepackage{diagbox}
\usepackage[labelformat=simple]{subcaption}
\usepackage{xcolor}

% configure imports
\DeclareMathOperator*{\argmin}{arg\,min}
\newcommand{\todo}[1]{\textcolor{red}{TODO: #1}}
\definecolor{darkgreen}{RGB}{46, 184, 46}
\newcommand{\half}{\frac{1}{2}}
\newcommand{\R}{\mathbb{R}}
\captionsetup[subfigure]{labelformat=empty, skip=0pt}
\captionsetup{labelsep=period}
\restylefloat{table}
\renewcommand\thesubfigure{(\alph{subfigure})}
\addto\captionsenglish{%
  \renewcommand{\figurename}{FIG.}%
  \renewcommand{\tablename}{TABLE}%
  \renewcommand{\appendixname}{APPENDIX}%
}

\title{Solid State}
\date{}

\begin{document}

\maketitle

\section*{Crystal Lattices}

\textbf{Bravais Lattice}
\begin{align}
L &= \{n_1 \vec{a}_1 + n_2 \vec{a}_2 + n_3 \vec{a}_3 \ \lvert \ n_1, n_2, n_3 \in \mathbb{Z}\}
\end{align}

\noindent
\textbf{Wigner-Seitz Unit Cell}
\begin{enumerate}
  \item Choose one lattice point. Draw lines to every other lattice point.
  \item Bisect each line with a plane normal to the line.
  \item Space bounded by planes forms the unit cell.
\end{enumerate}

\noindent
\textbf{Brillouin Zone} is Wigner-Seitz cell of reciprocal lattice.

\begin{align}
  B &= 2\pi A^{-T}
\end{align}

\end{document}
