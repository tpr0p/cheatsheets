\documentclass[12pt]{article}

% imports
\usepackage{appendix}
\usepackage{amsmath}
\usepackage{amssymb}
\usepackage[english]{babel}
\usepackage{bbold}
\usepackage[justification=centering]{caption}
\usepackage{float}
\usepackage[margin=1em]{geometry}
\usepackage{graphicx}
\usepackage{hyperref}
\usepackage[utf8]{inputenc}
\usepackage{layouts}
\usepackage{multirow}
\usepackage{multicol}
\usepackage{optidef}
\usepackage{physics}
\usepackage{setspace}
\usepackage{diagbox}
\usepackage[labelformat=simple]{subcaption}
\usepackage{xcolor}

% configure imports
\DeclareMathOperator*{\argmin}{arg\,min}
\newcommand{\todo}[1]{\textcolor{red}{TODO: #1}}
\definecolor{darkgreen}{RGB}{46, 184, 46}
\newcommand{\half}{\frac{1}{2}}
\newcommand{\R}{\mathbb{R}}
\captionsetup[subfigure]{labelformat=empty, skip=0pt}
\captionsetup{labelsep=period}
\restylefloat{table}
\renewcommand\thesubfigure{(\alph{subfigure})}
\addto\captionsenglish{%
  \renewcommand{\figurename}{FIG.}%
  \renewcommand{\tablename}{TABLE}%
  \renewcommand{\appendixname}{APPENDIX}%
}

\title{\vspace{-2em}EM}
\date{}

\begin{document}

\maketitle

\vspace{-4em}
\begin{multicols}{3}

  \textbf{Multipoles}
  \begin{align}
    q &= \int \rho(\vec{x}) d^3x\\
    \vec{p} &= \int \vec{x} \rho(\vec{x}) d^3x\\
    Q_{ij} &= \int (3x_ix_j-r^2\delta_{ij})\rho(\vec{x})d^3x\\ % Wald Ch. 2
    \vec{F}(\vec{x}) &= \int \rho(\vec{x}^{\prime} - \vec{x}) \vec{E}(\vec{x}^{\prime} - \vec{x}) d^3x^{\prime}\\
    &= \begin{aligned}[t]
      &q\vec{E}(\vec{x}^{\prime})\lvert_{\vec{x}} \ + \ (\vec{p} \cdot \vec{\nabla}_{\vec{x}^{\prime}}) \vec{E}(\vec{x}^{\prime}) \lvert_{\vec{x}}\\
      &+ \  \frac{1}{6} Q_{jk} \frac{\partial^2 \vec{E}(\vec{x}^{\prime})}{\partial x^{\prime}_j \partial x^{\prime}_k} \lvert_{\vec{x}} \ + \ \dots
    \end{aligned}\\
    \phi(\vec{x}) &= \frac{q}{\lvert \vec{x} \rvert} + \frac{\vec{p} \cdot \hat{x}}{{\lvert \vec{x} \rvert}^2} + \frac{1}{2} Q_{ij} \frac{\hat{x}_i \hat{x}_j}{{\lvert \vec{x} \rvert}^3} + \dots
  \end{align}
\end{multicols}

\end{document}
